\documentclass[a4paper,10pt]{article}
\usepackage[margin=1in]{geometry}
\usepackage{graphicx}
\usepackage{amsmath}
\usepackage{hyperref}

\title{Perceptron-Based Learning Tool: Project Proposal}
\author{
    S.A.R.R. Suraweeraarachchi (E/19/393) \\
    M.H.P. Thiwanka (E/19/407) \\
    L.M.L.M. Senevirathne (E/19/365) \\
    R.A.R.S Jayasekara (E/19/159) \\
    J.K. Wanasinghe (E/20/420)
}
\date{January 27, 2025}

\begin{document}

\maketitle

\section*{Course Title}
CO542 Neural Networks and Fuzzy Systems

\section*{Project Domain and Research Proposal}
This is the Project Domain and Research Proposal for Milestone 1.

\section{Goal}
The primary goal of this project is to identify a suitable domain and propose an application using neural networks. The focus is on developing a hardware-based learning tool to visualize the working of a perceptron.

\section{Problem Definition}
Understanding the internal workings of neural networks can be challenging, especially for beginners. Traditional software-based implementations do not provide an intuitive way to grasp the learning process. This project addresses the need for a tangible, hardware-based visualization tool that demonstrates how a perceptron updates its weights and makes decisions.

\section{Scope}
The project is designed as an educational tool for students and researchers interested in neural networks. It will focus on:

\begin{itemize}
    \item Implementing a single-layer perceptron with three numerical inputs.
    \item Using an Arduino-based system to represent weight adjustments in real-time.
    \item Visualizing learning through LED indicators and variable resistors.
    \item Providing a mobile and accessible demonstration setup for lecture halls and workshops.
\end{itemize}

\section{Justification for Using Perceptrons}
Perceptrons are the simplest form of artificial neural networks and serve as a foundation for more complex models. They are suitable for binary classification tasks and provide an effective way to demonstrate weight adjustments through hands-on interaction. Using a perceptron allows for real-time visualization, making it easier to understand core concepts like weighted summation, activation functions, and training iterations.

\section{Literature Review}
The perceptron algorithm, introduced by Rosenblatt in 1958, is one of the earliest machine learning models used for classification tasks. Traditional perceptron implementations are software-based, but hardware-based educational tools have been explored in recent years. 

Several research studies and open-source projects have demonstrated the feasibility of implementing perceptrons using microcontrollers like Arduino. These projects often use LEDs, voltage-sensitive components, and interactive visualizations to illustrate neural network concepts. Notable examples include:

\begin{itemize}
    \item Arduino-based Perceptron Simulations : These use microcontrollers to process inputs and activate LEDs based on learned weights.
    \item Analog Circuit Perceptrons : Resistor networks are used to model weight adjustments in learning.
    \item Educational AI Tools : Some projects integrate microcontrollers with software simulations to teach AI fundamentals.
\end{itemize}

\section{High-Level Design of the Proposed Model}

The perceptron model for this project follows a single-layer architecture with three main inputs. The architecture consists of:

\begin{itemize}
    \item \textbf{Input Layer}: Three numerical inputs (e.g., temperature, humidity, pressure)
    \item \textbf{Weights}: Represented by adjustable resistors or stored in an Arduino
    \item \textbf{Summation Function}: Computes weighted sum of inputs
    \item \textbf{Activation Function}: Step function (ON/OFF) or sigmoid function
    \item \textbf{Output Layer}: LED or display indicating classification
\end{itemize}

\subsection{Block Diagram}
\begin{center}
    % \includegraphics[width=0.8\textwidth]{perceptron_diagram.png}  % Placeholder for block diagram
\end{center}

The model will be implemented using an Arduino microcontroller to handle computations, while LEDs and resistors provide a physical visualization of learning. The system will train using predefined input-output pairs and adjust weights accordingly. 

\section{Conclusion}
This hardware-based perceptron visualization tool will serve as an educational aid for understanding fundamental neural network concepts. The implementation will include Arduino integration, LED-based weight representation, and interactive real-time training feedback.

\end{document}
